\documentclass{article}
\usepackage{amsmath}
\usepackage{txfonts}

\newcommand{\water}{\text{w}}
\newcommand{\oil}{\text{o}}
\newcommand{\gas}{\text{g}}

\title{Flux, Jacobian and Hessian of a three-phase subphysics}
\author{}
\date{}

\begin{document}
\maketitle

\section{Definitions}

%Let the suffices $\water$, $\oil$ and $\gas$ stand for water, oil and gas, respectively.

Let $s_\water$, $s_\oil$ and $s_\gas$ be the saturations of water, oil and gas, respectively.
All three saturations are related:
\begin{equation}\label{saturations}
s_\water + s_\oil + s_\gas = 1.
\end{equation}

Let $i$ be one of $\water$, $\oil$ or $\gas$. Then the following definitions apply:
Let $k_i$ stand for the {\em permeability}.
Let $\mu_i$ stand for the {\em viscosity}.
Let $\lambda_i = k_i/\mu_i$ stand for the {\em mobility} and let $\lambda = \lambda_\water + \lambda_\oil + \lambda_\gas$ be the {\em total mobility}.
Let $g_i$ stand for the {\em density}.
Let $v$ stand for the {\em velocity}.

\section{Flux}
The flux of a three-phase subphysics is of the form:
\begin{equation}\label{flux}
F(s_\water, s_\oil) = \left(
\begin{array}{c}
f_\water \\
f_\oil
\end{array}
\right).
\end{equation}

\subsection{$\boldsymbol{f_\water}$}\label{water_flux}

The part of the flux that corresponds to the water is defined as:
\begin{equation}\label{fw}
f_\water = \frac{\lambda_\water}{\lambda} \bigl( v + \lambda_\oil(g_\water - g_\oil) + \lambda_\gas(g_\water - g_\gas) \bigr).
\end{equation}

\subsection{$\boldsymbol{f_\oil}$}\label{oil_flux}

The part of the flux that corresponds to the water is defined as:
\begin{equation}\label{fw}
f_\oil = \frac{\lambda_\oil}{\lambda} \bigl( v + \lambda_\water(g_\oil - g_\water) + \lambda_\gas(g_\oil - g_\gas) \bigr).
\end{equation}

\section{Jacobian}

\subsection{$\boldsymbol{\partial f_\water/\partial s_\water}$}\label{dfw_dsw}



\end{document}

